\documentclass[class=report, float=false, crop=false]{standalone}

../presentation_7_30_18/preamble.tex

\graphicspath{{conclusion}}

\begin{document}

\chapter*{Conclusion}
\label{conclusion}
\addcontentsline{toc}{chapter}{Conclusion}

We have numerically studied a minimal model of active Brownian disks, which perform independent persistent random walks, with purely repulsive interparticle potential.\\

This model system displays motility-induced phase separation, a common feature of systems of self-propelled particles. We have showed that, when increasing the persistence time of the random walk of each particle, this phenomenon, in which the system spontaneously separates into a dense fluid phase of lesser motility and an active gas phase of greater motility, is accompanied by an increase of dynamic heterogeneity as measured by the cooperativity of displacements and a decrease in the directionality of the displacement correlations.\\

Nonetheless, such observations could not be made when increasing the activity to the point of phase separation by simply increasing the particles' velocities. We suggest to investigate other paths through the phase diagram to rigorously establish the relevance of each parameter in this transition.\\

We have also brought evidence that, in the phase separated regime, our model system develops scale-free shear strain correlations, a common feature of glass-forming material. In the fluid regime though, they decay exponentially, as in liquids above the melting point. Our current efforts are dedicated to characterising the transition from the fluid regime to the phase separated regime as seen by the decay of these correlations.

\end{document}
