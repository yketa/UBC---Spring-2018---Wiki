\documentclass[class=report, float=false, crop=false]{standalone}

../presentation_7_30_18/preamble.tex

\graphicspath{{figures/images/}{figures/figs/}}

\begin{document}

\chapter{Shear strain correlation}
\label{chap:strain}

\section{Plastic deformation}

% explain the origin of plastic flow, the build up of shear strain correlations, their characteristics

Current theories of plastic deformation of amorphous solids rely on the existence of \textit{shear transformation zones}, which are special sites, a few particles wide, where particles are able to rearrange themselves in response to applied stresses \cite{falk1998dynamics}. Plastic deformation is then the consequence of localized irreversible rearrangements, which are coupled by elastic strain fields \cite{nicolas2014spatiotemporal}.

\section{Shear strain}

\subsection{Shear strain}

\subsection{Shear strain correlation}

\section{Real space method}

\subsection{Method}

\subsection{Results}

\section{Collective mean square displacement method}

\subsection{Collective mean square displacement}

StackExchange \faStackExchange~ \cite{stackexchange}

\subsection{Results}

\end{document}
